%! TEX root = ../note-kdre-inference.tex

\section{Density Estimation}

\begin{assumption}
\label{asm--kde-dgp}
\(X, X_{1}, \dots, X_{n}\) are iid random \(\mathbb{R}^{d}\)-vectors all with
Lebesgue density \(f\).
\end{assumption}

\begin{assumption}
\label{asm--kernel}
\(K : \mathbb{R}^{d} \to \mathbb{R}\) be a Lebesgue-integrable function such
that
\begin{enumerate}[label=(\roman*)]
  \item
    \label{asm--kernel-unit-integral}
    \(\int K (u) \; \mathrm{d} u = 1\)
  \item \label{asm--kernel-thin-tails}
    \(\lim_{\|u\| \to \infty} \|u\|^{d} K (u) = 0\).
\end{enumerate}
\end{assumption}

For a function \(K : \mathbb{R}^{d} \to \mathbb{R}\) satisfying
\Cref{asm--kernel}, define
\begin{equation}
  K_{h} (u) := \frac{1}{h^{d}} K \left( \frac{u}{h} \right).
  \label{eqn--kernel-Kh}
\end{equation}
The kernel density estimator and its mean are
\begin{equation}
  \widehat{f}_{n, h} (x) = \frac{1}{n} \sum_{i = 1}^{n} K_{h} \left( X_{i} - x
  \right) \quad \text{and} \quad
  f_{h} (x) = \E \left[ K_{h} (X - x) \right] = \int f (x + u) K_{h} (u) \;
  \mathrm{d} u.
  \label{eqn--kde-and-mean}
\end{equation}
Note that the final equality in \eqref{eqn--kde-and-mean} follows from the usual
change of variables formula.

\begin{lemma}
\label{lem--kernel-polynomial-mean}
Let \(h, p \in (0, \infty)\) and \(x \in \mathbb{R}^{d}\).
Let \(K : \mathbb{R}^{d} \to \mathbb{R}\) and \(g : \mathbb{R}^{d} \to
\mathbb{R}\) be measurable functions such that
\(\int |g (x + v h)| |K (v)|^{p} \; \mathrm{d} v < \infty\).
Then
\begin{equation}
  \int g (x + u) K (u / h)^{p} \; \mathrm{d} u = h^{d} \int g (x + v h) K
  (v)^{p} \; \mathrm{d} v.
\end{equation}
\end{lemma}

\begin{proof}[Proof of \Cref{lem--kernel-polynomial-mean}]
Use the change of variables \(v = u / h\).
\end{proof}

Note that
\begin{equation*}
  \E \left[ K ((X - x) / h)^{p} \right] = \int f (\xi) K ((\xi - x) / h)^{p} \;
  \mathrm{d} \xi
\end{equation*}
and by the change of variables \(u = (\xi - x) / h\), so that \(\xi = x + u h\),
\begin{equation*}
  \E \left[ K ((X - x) / h)^{p} \right] = h^{d} \int f (x + u h) K (u)^{p} \;
  \mathrm{d} u.
\end{equation*}
Therefore,
\begin{align*}
  \E \left[ K_{h} (X - x)^{p} \right] =
  & \, \frac{1}{h^{d p}} = \E \left[ K ((X - x) / h)^{p} \right] =
  \frac{h^{d}}{h^{d p}} \int f (x + u h) K (u)^{p} \; \mathrm{d} u \\
  =
  & \, \frac{1}{h^{d (p - 1)}} \int f (x + u h) K (u)^{p} \; \mathrm{d} u.
\end{align*}

Let
\begin{equation*}
  c_{n, h}^{2} = n \Var \left[ K_{h} (X - x) \right] \quad b_{n, h}^{2 + v} = n
  \E \left[ \left| K_{h} (X - x) - \E \left[ K_{h} (X - x) \right] \right|^{2 +
  v} \right].
\end{equation*}
Then
\begin{align*}
  c_{n, h}^{2} =
  & \, n \left( \E \left[ K_{h} (X - x)^{2} \right] - \E \left[ K_{h} (X - x)
  \right]^{2} \right) \\
  =
  & \, n \left( \frac{1}{h^{d}} \int f (x + u h) K (u)^{2} \; \mathrm{d} u -
  \left[ \int f (x + u h) K (u) \; \mathrm{d} u \right]^{2} \right) \\
  =
  & \, \frac{n}{h^{d}} \left( \int f (x + u h) K (u)^{2} \; \mathrm{d} u + O
  \left( h^{d} \right) \right).
\end{align*}
For any real numbers \(a, b, p\) with \(p \geq 1\), \(|a + b|^{p} \leq 2^{p - 1}
\left( |a|^{p} + |b|^{p} \right)\).
Therefore,
\begin{align*}
  b_{n, h}^{2 + v} \leq
  & \, 2^{1 + v} n \left( \E \left[ \left| K_{h} (X - x) \right|^{2 + v} \right]
  + \left| \E \left[ K_{h} (X - x) \right] \right|^{2 + v} \right) \\
  =
  & \, 2^{1 + v} n \left( \frac{1}{h^{d (1 + v)}} \int f (x + u h) |K (u)|^{2 +
  v} \; \mathrm{d} u + \left| \int f (x + u h) K (u) \; \mathrm{d} u \right|^{2
  + v} \right) \\
  =
  & \, \frac{2^{1 + v} n}{h^{d (1 + v)}} \left( \int f (x + u h) |K (u)|^{2 + v}
  \; \mathrm{d} u + O \left( h^{d (1 + v)} \right) \right).
\end{align*}
Then,
\begin{align*}
  \frac{b_{n, h}^{2 + v}}{c_{n, h}^{2 + v}} \leq
  & \, \frac{\frac{2^{1 + v} n}{h^{d (1 + v)}} \left( \int f (x + u h) |K
  (u)|^{2 + v} \; \mathrm{d} u + O \left( h^{d (1 + v)} \right) \right)}{\left(
  \frac{n}{h^{d}} \left( \int f (x + u h) K (u)^{2} \; \mathrm{d} u + O \left(
  h^{d} \right) \right) \right)^{1 + (v / 2)}} \\
  =
  & \, \frac{1}{\left( n h^{d} \right)^{v / 2}} \cdot \frac{2^{1 + v} \left(
  \int f (x + u h) |K (u)|^{2 + v} \; \mathrm{d} u + O \left( h^{d (1 + v)}
  \right) \right)}{\left( \int f (x + u h) K (u)^{2} \; \mathrm{d} u + O \left(
  h^{d} \right) \right)^{1 + (v / 2)}}
\end{align*}
So Liapunov's condition holds as long as \(\lim 1 / \left( n h^{d} \right) =
0\).

\begin{theorem}
\label{thm--kde-asymptotic-normality}
Let \(\left\{ h_{n} \right\}\) be a real sequence such that \(h_{n} > 0\) for
every \(n \in \mathbb{N}\), \(\lim_{n \to \infty} h_{n} = 0\), and \(\lim_{n \to
\infty} 1 / \left( n h_{n}^{d} \right) = 0\).
Let \(K\) be a function satisfying \Cref{asm--convl-kernel} and let the
probability density \(f\) in \Cref{asm--kde-dgp} be continuous at \(x \in
\mathbb{R}^{d}\).
Suppose that for some \(v > 0\), \(\E \left[ \left| K \left( (X - x) / h_{n}
\right) \right|^{2 + v} \right] < \infty\) for all \(n \in \mathbb{N}\).
Then
\begin{equation*}
  \sqrt{n h_{n}^{d}} \left( \widehat{f}_{n, h_{n}} (x) - f_{h_{n}} (x) \right)
  \rightsquigarrow \N \left( 0, f (x) \int K (u)^{2} \; \mathrm{d} u \right).
\end{equation*}
\end{theorem}

\begin{proof}[Proof of \Cref{thm--kde-asymptotic-normality}]
Define
\begin{equation*}
  W_{i, h} (x) = K_{h} \left( X_{i} - x \right) - \E \left[ K_{h} (X - x)
  \right], \quad \text{and} \quad c_{n, h} (x)^{2} = \sum_{i = 1}^{n} \Var
  \left[ W_{i, h} (x) \right].
\end{equation*}
Since \(W_{1, h} (x), \dots, W_{n, h} (x)\) are iid,
\begin{align*}
  c_{n, h} (x)^{2} =
  & \, n \left( \E \left[ K_{h} \left( X - x \right)^{2} \right] - \E \left[
  K_{h} (X - x) \right]^{2} \right) \\
  =
  & \, n \left( \frac{1}{h^{d}} \int f (x + u h) K (u)^{2} \; \mathrm{d} u -
  \left[ \int f (x + u h) K (u) \; \mathrm{d} u \right]^{2} \right) \\
  =
  & \, \frac{n}{h^{d}} \left( \int f (x + u h) K (u)^{2} \; \mathrm{d} u -
  h^{d} \left[ \int f (x + u h) K (u) \; \mathrm{d} u \right]^{2} \right).
\end{align*}
Hence,
\begin{equation*}
  \frac{c_{n, h} (x)^{2} h^{d}}{n} = \int f (x + u h) K (u)^{2} \; \mathrm{d} u
  - h^{d} \left[ \int f (x + u h) K (u) \; \mathrm{d} u \right]^{2}
\end{equation*}
By \(h = h_{n} \to 0\) and dominated convergence,
\begin{equation*}
  \frac{c_{n, h} (x)^{2} h^{d}}{n} \to f (x) \int K (u)^{2} \; \mathrm{d} u.
\end{equation*}
Since \(1 / \left( n h^{d} \right) \to 0\), by Liapunov's CLT
\begin{align*}
  \sqrt{n h^{d}} \left( \widehat{f}_{n, h} (x) - f_{h} (x) \right) =
  & \, \sqrt{\frac{h^{d}}{n}} \cdot \sum_{i = 1}^{n} W_{i, h} (x) \\
  =
  & \, \sqrt{\frac{c_{n, h} (x)^{2} h^{d}}{n}} \cdot \frac{\sum_{i = 1}^{n}
  W_{i, h} (x)}{c_{n, h} (x)} \\
  \rightsquigarrow
  & \, \sqrt{f (x) \int K (u)^{2} \; \mathrm{d} u} \cdot \N (0, 1) \\
  =
  & \, \N \left( 0, f (x) \int K (u)^{2} \; \mathrm{d} u \right).
\end{align*}
\end{proof}

Let \(\widehat{V}_{n, h} (x)\) and \(V_{h} (x)\) be defined by
\begin{equation}
  \begin{split}
  \widehat{V}_{n, h} (x) =
  & \, \frac{1}{n} \sum_{i = 1}^{n} K_{h} \left( X_{i} - x \right)^{2} - \left(
  \frac{1}{n} \sum_{i = 1}^{n} K_{h} \left( X_{i} - x \right) \right)^{2} \\
  =
  & \, \frac{1}{n} \sum_{i = 1}^{n} K_{h} \left( X_{i} - x \right)^{2} -
  \widehat{f}_{n, h} (x)^{2}, \\
  \text{and} \quad
  V_{h} (x) =
  & \, \E \left[ K_{h} (X - x)^{2} \right] - \E \left[ K_{h} (X - x) \right]^{2}
  = \E \left[ K_{h} (X - x)^{2} \right] - f_{h} (x)^{2}
  \end{split}
\end{equation}
Furthermore, set
\begin{equation*}
  \widehat{Z}_{n} (x) = \frac{\widehat{f}_{n, h_{n}} (x) - f_{h}
  (x)}{\sqrt{\widehat{V}_{n, h_{n}} (x) / n}} \quad \text{and} \quad Z_{n} (x) =
  \frac{\widehat{f}_{n, h_{n}} (x) - f_{h} (x)}{\sqrt{V_{h_{n}} (x) / n}}.
\end{equation*}
We wish to show that
\begin{equation*}
  \text{If }
  \lim_{n \to \infty} \max \left\{ h_{n}, \frac{1}{n h_{n}^{d}} \right\} = 0,
  \text{ then } \widehat{Z}_{n} (x) \rightsquigarrow \N (0, 1).
\end{equation*}
But we know that if the premise of the above condition holds, that \(Z_{n} (x)
\rightsquigarrow \N (0, 1)\) and so, by Slutsky's Theorem, we are left to show
that
\begin{equation*}
  \frac{\widehat{V}_{n, h} (x)}{V_{h} (x)} \overset{\mathrm{p}}{\to} 1,
\end{equation*}
or equivalently that
\begin{equation*}
  \widehat{V}_{n, h} (x) - V_{h} (x) = o_{\mathrm{p}} \left( V_{h} (x) \right).
\end{equation*}
To that end,
\begin{equation*}
  \widehat{V}_{n, h} (x) - V_{h} (x) = \frac{1}{n} \sum_{i = 1}^{n} \left\{
  K_{h} \left( X_{i} - x \right)^{2} - \E \left[ K_{h} (X - x)^{2} \right]
  \right\} + f_{h} (x)^{2} - \widehat{f}_{n, h} (x)^{2}
\end{equation*}
Hence, we can show the following two conditions separately:
\begin{align}
  \frac{1}{n} \sum_{i = 1}^{n} \left\{
  K_{h} \left( X_{i} - x \right)^{2} - \E \left[ K_{h} (X - x)^{2} \right]
  \right\} =
  & \,  o_{\mathrm{p}} \left( V_{h} (x) \right) \\
  \widehat{f}_{n, h} (x)^{2} - f_{h} (x)^{2} =
  & \, o_{\mathrm{p}} \left( V_{h} (x)
  \right)
\end{align}
Start with the latter and write
\begin{equation*}
  \widehat{f}_{n, h} (x)^{2} - f_{h} (x)^{2} = \left( \widehat{f}_{n, h} (x) +
  f_{h} (x) \right) \left( \widehat{f}_{n, h} (x) - f_{h} (x) \right).
\end{equation*}
Recall that \(\widehat{f}_{n, h} (x) - f_{h} (x) = O_{p} \left( \sqrt{V_{h} (x)
/ n} \right)\).
A sufficient condition is then \(1 / n = o \left( V_{h} (x) \right)\).
To show this, recall that
\begin{equation*}
  V_{h} (x) = \frac{1}{h^{d}} \left( \int f (x + u h) K (u)^{2} \; \mathrm{d} u
  - h^{d} \left[ \int f (x + u h) K (u) \; \mathrm{d} u \right]^{2} \right).
\end{equation*}
So, as long as \(f (x) > 0\),
\begin{equation*}
  \frac{1}{n V_{h} (x)} = \frac{h^{d}}{n \left( \int f (x + u h) K
  (u)^{2} \; \mathrm{d} u - h^{d} \left[ \int f (x + u h) K (u) \; \mathrm{d} u
  \right]^{2} \right)} \to 0.
\end{equation*}

Now set
\begin{equation*}
  D_{n, h} (x) = \frac{1}{n} \sum_{i = 1}^{n} \left\{ K_{h} \left( X_{i} - x
  \right)^{2} - \E \left[ K_{h} (X - x)^{2} \right] \right\}.
\end{equation*}
We know that \(D_{n, h} (x) = O_{\mathrm{p}} \left( \Var \left[ D_{n, h} (x)
\right]^{1 / 2} \right)\) and so we wish to show that
\begin{equation*}
  \Var \left[ D_{n, h} (x) \right]^{1 / 2} = o \left( V_{h} (x) \right) \text{
  or equivalently } \Var \left[ D_{n, h} (x) \right] = o \left( V_{h}
  (x)^{2} \right).
\end{equation*}
But
\begin{align*}
  \Var \left[ D_{n, h} (x) \right] =
  & \, \frac{\E \left[ K_{h} (X - x)^{4} \right] - \E \left[ K_{h} (X - x)^{2}
  \right]^{2}}{n} = \frac{\E \left[ K_{h} (X - x)^{4} \right]}{n} - \frac{O
  \left( V_{h} (x)^{2} \right)}{n} \\
  =
  & \, \frac{\E \left[ K_{h} (X - x)^{4} \right]}{n} - o \left( V_{h} (x)^{2}
  \right) \\
  =
  & \, \frac{\frac{1}{h^{3 d}} \int f (x + u h) K (u)^{4} \; \mathrm{d} u}{n} -
  o \left( V_{h} (x)^{2} \right)
\end{align*}
We are done if the first term is \(o \left( V_{h} (x)^{2} \right)\)
\begin{align*}
  \frac{\frac{\frac{1}{h^{3 d}} \int f (x + u h) K (u)^{4} \; \mathrm{d}
  u}{n}}{V_{h} (x)^{2}} =
  & \, \frac{\int f (x + u h) K (u)^{4} \; \mathrm{d} u}{n h^{3 d} V_{h}
  (x)^{2}} \\
  =
  & \, \frac{\int f (x + u h) K (u)^{4} \; \mathrm{d} u}{n h^{3 d} \left(
  \frac{1}{h^{d}} \left( \int f (x + u h) K (u)^{2} \; \mathrm{d} u - h^{d}
  \left[ \int f (x + u h) K (u) \; \mathrm{d} u \right]^{2} \right) \right)^{2}}
  \\
  =
  & \, \frac{h^{d} \int f (x + u h) K (u)^{4} \; \mathrm{d} u}{n \left(
  \left( \int f (x + u h) K (u)^{2} \; \mathrm{d} u - h^{d}
  \left[ \int f (x + u h) K (u) \; \mathrm{d} u \right]^{2} \right) \right)^{2}}
  \\
  =
  & \, o (1).
\end{align*}


%%% Local Variables:
%%% mode: LaTeX
%%% TeX-master: "../note-kdre-inference"
%%% End:
