%! TEX root = ../note-kdre-inference.tex

\section{Approximation Results for Kernel Estimators}
% \label{sec--kernel-approx}

Here, we consider approximation by convolution.
The exposition here is mainly based on
\citet[pp. 362--365]{1999paganNonparametricEconometrics}.%
\footnote{In my PDF copy of \citet{1999paganNonparametricEconometrics}, this
corresponds to pp. 381--384 (in PDF).}
I should note that \citet{1999paganNonparametricEconometrics} borrow from
\citet{1962parzenEstimationProbabilityDensity} who in turn borrows from
\citet{1955bochnerHarmonicAnalysisTheory}.

\begin{assumption}
\label{asm--convl-kernel}
\(\left\{ \kappa_{h} : h \in (0, \infty) \right\}\) is a family of
functions mapping \(\mathbb{R}^{d}\) to \(\mathbb{R}\)
satisfying the following.
\begin{enumerate}[label=(\roman*)]
  \item \label{asm--convl-kernel-L1}
    For each \(h \in (0, \infty)\), \(\kappa_{h}\) is integrable with
    \(\int \left| \kappa_{h} (u) \right| \; \mathrm{d} u =:
    \overline{\kappa}_{h} < \infty\).
  \item \label{asm--convl-kernel-integral-unif-bound}
    There is a \(h_{\ast} \in (0, \infty)\) and \(\overline{\kappa} \in (0,
    \infty)\) such that for every \(h \in \left( 0, h_{\ast} \right]\),
    \(\overline{\kappa}_{h} \leq \overline{\kappa}\).
  \item \label{asm--convl-kernel-conv-prob}
    For each \(\delta \in (0, \infty)\), \(\lim_{h \to 0} \int_{\|u\| \geq
    \delta} \left| \kappa_{h} (u) \right| \mathrm{d} u = 0\).
  \item \label{asm--convl-kernel-integral-lim}
    For each \(h \in (0, \infty)\), \(\int \kappa_{h} (u) \; \mathrm{d} u =
    \underline{\kappa}_{h}\) and \(\lim_{h \to 0} \underline{\kappa}_{h} =
    \underline{\kappa} \in \mathbb{R}\).
\end{enumerate}
\end{assumption}

\begin{remark}
% \label{rem--convl-kernel-non-neg}
If the family \(\left\{ \kappa_{h} \right\}\) satisfies \(\kappa_{h} \geq 0\)
almost everywhere for each \(h \in (0, \infty)\), then \Cref{asm--convl-kernel}
\ref{asm--convl-kernel-L1} and \ref{asm--convl-kernel-integral-lim} imply
\ref{asm--convl-kernel-integral-unif-bound}, since in this case
\(\underline{\kappa}_{h} = \overline{\kappa}_{h}\) for every \(h \in (0,
\infty)\) and we can choose \(h_{\ast}\) to satisfy \(\left|
\underline{\kappa}_{h} - \underline{\kappa}
\right| \leq 1\) for every \(h \in \left( 0, h_{\ast} \right]\).
Then set \(\overline{\kappa} = \left| \underline{\kappa} \right| + 1\).
\end{remark}

\begin{lemma}
\label{lem--convl-kernel-eg}
Let \(\kappa : \mathbb{R}^{d} \to \mathbb{R}\) be an integrable function.
Define \(\kappa_{h} (u) := h^{- d} \kappa (u / h)\).
Then \(\left\{ \kappa_{h} \right\}\) satisfies \Cref{asm--convl-kernel}.
\end{lemma}

\begin{proof}[Proof of \Cref{lem--convl-kernel-eg}]
By the change of variables \(v = u / h\),
\begin{align*}
  & \underline{\kappa}_{h} = \int \kappa_{h} (u) \; \mathrm{d} u = \int h^{- d}
  \kappa (u / h) \; \mathrm{d} u = \int \kappa (v) \; \mathrm{d} v \quad \forall
  h \in (0, \infty), \\
  & \overline{\kappa}_{h} = \int \left| \kappa_{h} (u) \right| \; \mathrm{d} u =
  \int h^{- d} |\kappa (u / h)| \; \mathrm{d} u = \int |\kappa (v)| \;
  \mathrm{d} v =: \overline{\kappa},
  \\
  & \int_{\|u\| \geq \delta} \left| \kappa_{h} (u) \right| \; \mathrm{d} u =
  \int_{\|u\| \geq \delta} h^{- d} \left| \kappa (u / h) \right| \; \mathrm{d} u
  = \int_{\|v\| \geq \delta / h} |\kappa (v)| \; \mathrm{d} v \to 0 \\
  & \text{since} \quad \{v : \|v\| \geq \delta / h\} \searrow \emptyset, \quad
  \text{as} \quad h \to 0.
\end{align*}
\end{proof}

For a measurable function \(g : \mathbb{R}^{d} \to \mathbb{R}\),
consider the convolution based approximant
\begin{equation}
  \begin{gathered}
  g_{h} (x) := T_{h} [g] (x) := \int g (x + u) \kappa_{h} (u) \; \mathrm{d} u \\
  \text{for a family } \left\{ \kappa_{h} \right\} \text{ satisfying
  \Cref{asm--convl-kernel}.}
  \end{gathered}
  \label{eqn--convl-kernel-approximant}
\end{equation}
Assume henceforth that for the point \(x\) in question, the integral in
\eqref{eqn--convl-kernel-approximant} exists and is finite.
The natural question of primitive assumptions for this existence and finiteness.
We first provide a pointwise error bound for \(g_{h} - g\).

\begin{theorem}
\label{thm--convl-kernel-point-error-bound}
Let \(\left\{ \kappa_{h} \right\}\) satisfy \Cref{asm--convl-kernel}
\ref{asm--convl-kernel-L1}-\ref{asm--convl-kernel-conv-prob}.
Let \(h, \delta \in (0, \infty)\), \(x \in \mathbb{R}^{d}\), and let
\(g : \mathbb{R}^{d} \to \mathbb{R}\) be a measurable function.
Assume the integral in \eqref{eqn--convl-kernel-approximant} exists and is
finite.
Then
\begin{equation}
  \begin{split}
    \left| g_{h} (x) - g (x) \cdot \underline{\kappa}_{h} \right| \leq
    & \, \left\{ \sup_{\|u\| < \delta} |g (x + u) - g (x)| \right\} \cdot
    \overline{\kappa} + \int_{\|u\| \geq \delta} |g (x + u)| \cdot \left|
    \kappa_{h} (u) \right| \; \mathrm{d} u \\
    & + |g (x)| \int_{\|u\| \geq \delta} \left| \kappa_{h} (u) \right| \;
    \mathrm{d} u.
  \end{split}
  \label{eqn--convl-kernel-inequality-main}
\end{equation}
\end{theorem}

\begin{proof}[Proof of \Cref{thm--convl-kernel-point-error-bound}]
\begin{equation*}
  \left| g_{h} (x) - g (x) \cdot \underline{\kappa}_{h} \right| = \left| \int (g
  (x + u) - g (x)) \kappa_{h} (u) \; \mathrm{d} u \right| \leq \int |g (x + u) -
  g (x)| \left| \kappa_{h} (u) \right| \; \mathrm{d} u.
\end{equation*}
Split the integral on the right:
\begin{align*}
  \left| g_{h} (x) - g (x) \cdot \underline{\kappa}_{h} \right| \leq
  & \, \int_{\|u\| < \delta} |g (x + u) - g (x)| \left| \kappa_{h} (u) \right|
  \; \mathrm{d} u \\
  & + \int_{\|u\| \geq \delta} |g (x + u) - g (x)| \left| \kappa_{h} (u) \right|
  \; \mathrm{d} u \\
  \leq
  & \, \left[ \sup_{\|u\| < \delta} |g (x + u) - g (x)| \right] \int \left|
  \kappa_{h} (u) \right| \; \mathrm{d} u \\
  & + \int_{\|u\| \geq \delta} |g (x + u)| \left| \kappa_{h} (u) \right| \;
  \mathrm{d} u + |g (x)| \int_{\|u\| \geq \delta} \left| \kappa_{h} (u) \right|
  \; \mathrm{d} u.
\end{align*}
In the last inequality above, bound the integral in the first term by
\(\overline{\kappa}\) to get \eqref{eqn--convl-kernel-inequality-main}.
\end{proof}

\begin{lemma}
\label{lem--convl-kernel-conv-criteria}
Let \(\left\{ \kappa_{h} \right\}\) satisfy
\Cref{asm--convl-kernel}
\ref{asm--convl-kernel-L1}-\ref{asm--convl-kernel-conv-prob}.
Let \(x \in \mathbb{R}^{d}\), and let \(g : \mathbb{R}^{d} \to \mathbb{R}\) be a
measurable function that is continuous at \(x\).
Assume the integral in \eqref{eqn--convl-kernel-approximant} exists and is
finite for \(h > 0\) sufficiently small.
\begin{equation}
  \begin{gathered}
  \text{If} \quad \forall \delta > 0, \quad \lim_{h \to 0} \int_{\|u\| \geq
  \delta} |g (x + u)| \left| \kappa_{h} (u) \right| \; \mathrm{d} u = 0, \quad
  \text{then} \\
  \lim_{h \to \infty} \left( g_{h} (x) - g (x) \cdot \underline{\kappa}_{h}
  \right) = 0.
\end{gathered}
  \label{eqn--convl-kernel-conv-criteria}
\end{equation}
Furthermore if \(\left\{ \kappa_{h} \right\}\) also satisfies satisfies
\Cref{asm--convl-kernel} \ref{asm--convl-kernel-integral-lim} then the premise
in \eqref{eqn--convl-kernel-conv-criteria} also implies
\begin{equation}
  \lim_{h \to \infty} g_{h} (x) = g (x) \underline{\kappa}.
  \label{eqn--convl-kernel-conv-main}
\end{equation}
\end{lemma}

\begin{proof}[Proof of \Cref{lem--convl-kernel-conv-criteria}]
Under \Cref{asm--convl-kernel} \ref{asm--convl-kernel-integral-lim},
\eqref{eqn--convl-kernel-conv-main} is an immediate consequence of
\eqref{eqn--convl-kernel-conv-criteria}.
So we prove \eqref{eqn--convl-kernel-conv-criteria}.
Since \(g\) is continuous at \(x\), the first term in
\eqref{eqn--convl-kernel-inequality-main} can be controlled by choice of
\(\delta\) sufficiently small.
For any choice of \(\delta\), the second term in
\eqref{eqn--convl-kernel-inequality-main} is controlled by choice of \(h\)
sufficiently small by
\eqref{eqn--convl-kernel-conv-criteria}.
For any choice of \(\delta\), the third term in
\eqref{eqn--convl-kernel-inequality-main} is controlled by choice of \(h\)
sufficiently small by \Cref{asm--convl-kernel}
\ref{asm--convl-kernel-conv-prob}.
\end{proof}

We therefore need some way(s) to show \eqref{eqn--convl-kernel-conv-criteria} to
deal with the second term in \eqref{eqn--convl-kernel-inequality-main}.
We present two ways to do this.

\begin{theorem}
\label{thm--convl-kernel-conv-bounded}
Let \(\left\{ \kappa_{h} \right\}\) satisfy \Cref{asm--convl-kernel}
\ref{asm--convl-kernel-L1}-\ref{asm--convl-kernel-conv-prob}.
Let \(x \in \mathbb{R}^{d}\), and \(g : \mathbb{R}^{d} \to \mathbb{R}\)
be a bounded and measurable function that is continuous at \(x\).
Then the integral in \eqref{eqn--convl-kernel-approximant} exists and is finite
and furthermore, \(\lim_{h \to 0} \left( g_{h} (x) - g (x)
\underline{\kappa}_{h} \right) = 0\).
If in addition, \(\left\{ \kappa_{h} \right\}\) satisfies
\Cref{asm--convl-kernel} \ref{asm--convl-kernel-integral-lim}, then
\(\lim_{h \to 0} g_{h} (x) = g (x) \underline{\kappa}\)
\end{theorem}

\begin{proof}[Proof of \Cref{thm--convl-kernel-conv-bounded}]
Existence and finiteness of the integral in
\eqref{eqn--convl-kernel-approximant} are immediate consequences of the
hypothesis of \(g\) being bounded.
Furthermore by this hypothesis, given any \(\delta > 0\)
\begin{equation*}
  \int_{\|u\| \geq \delta} |g (x + u)| \left| \kappa_{h} (u) \right| \;
  \mathrm{d} u \leq \sup_{y \in \mathbb{R}^{d}} |g (y)| \int_{\|u\| \geq \delta}
  \left| \kappa_{h} (u) \right| \; \mathrm{d} u \to 0 \quad \text{as} \quad h
  \to 0.
\end{equation*}
Both limiting claims in \Cref{thm--convl-kernel-conv-bounded} now follow from
\eqref{eqn--convl-kernel-conv-criteria} in
\Cref{lem--convl-kernel-conv-criteria}.
\end{proof}

\begin{assumption}
\label{asm--convl-kernel-conv-thin-tail}
The family \(\left\{ \kappa_{h} \right\}\) in
\Cref{asm--convl-kernel} also satisfies the following condition:
\begin{equation}
  \forall \delta > 0, \quad
  \lim_{h \to 0} \sup_{u \in \mathbb{R}^{d} : \|u\| \geq \delta} \|u\|^{d}
  \kappa_{h} (u) = 0.
  \label{eqn--conv-kernel-thin-tail-condition}
\end{equation}
\end{assumption}

\begin{lemma}
\label{lem--convl-kernel-eg-thin-tail}
Consider the function \(\kappa\) and the associated family \(\left\{ \kappa_{h}
\right\}\) in \Cref{lem--convl-kernel-eg}.
\begin{equation}
  \text{If } \lim_{\|y\| \to \infty} \|y\|^{d} \kappa (y) = 0, \text{ then }
  \left\{ \kappa_{h} \right\} \text{ satisfies
  \Cref{asm--convl-kernel-conv-thin-tail}.}
  \label{eqn--conv-kernel-thin-tail-condition-1}
\end{equation}
\end{lemma}

\begin{proof}[Proof of \Cref{lem--convl-kernel-eg-thin-tail}]
Note that \(\sup_{\|u\| \geq \delta} \|u\|^{d} \kappa_{h} (u) = \sup_{\|u\| \geq
\delta} \| u / h \|^{d} \kappa (u / h)\).
Given any \(\delta > 0\), the premise in
\eqref{eqn--conv-kernel-thin-tail-condition-1} ensures that the right hand side
can be made arbitrarily small when \(h \to 0\).
Thus \eqref{eqn--conv-kernel-thin-tail-condition} is satisfied.
\end{proof}

\begin{theorem}
\label{thm--convl-kernel-conv-thin-tail}
Suppose the family \(\left\{ \kappa_{h} \right\}\) satisfies
\Cref{asm--convl-kernel}
\ref{asm--convl-kernel-L1}-\ref{asm--convl-kernel-conv-prob} and
\Cref{asm--convl-kernel-conv-thin-tail}.
Let \(x \in \mathbb{R}^{d}\), and let \(g : \mathbb{R}^{d} \to \mathbb{R}\) be
an integrable function that is continuous at \(x\).
Then for \(h > 0\) sufficiently small, the integral in
\eqref{eqn--convl-kernel-approximant} exists and is finite.
In addition, \(\lim_{h \to 0} \left( g_{h} (x) - g (x) \underline{\kappa}_{h}
\right) = 0\).
Furthermore, if \(\left\{ \kappa_{h} \right\}\) satisfies
\Cref{asm--convl-kernel}
\ref{asm--convl-kernel-integral-lim} then \(\lim_{h \to 0} g_{h} (x) = g (x)
\underline{\kappa}\).
\end{theorem}

\begin{proof}[Proof of \Cref{thm--convl-kernel-conv-thin-tail}]
For existence and finiteness of the integral in
\eqref{eqn--convl-kernel-approximant}, first note that for any \(\delta > 0\),
\begin{equation*}
  |g (x + u)| \left| \kappa_{h} (u) \right| \leq |g (x + u)| \left| \kappa_{h}
  (u) \right|
  \cdot \mathbf{1} \left\{ \|u\| < \delta \right\} + |g (x + u)| \left|
  \kappa_{h} (u) \right| \cdot \mathbf{1} \left\{ \|u\| \geq \delta \right\}.
\end{equation*}
Take any \(\varepsilon \in (0, \infty)\) and choose \(\delta :=
\delta_{\varepsilon, x} \in (0, \infty)\) to ensure that \(|g (x + u) - g (x)| <
\varepsilon\) for \(\|u\| < \delta\).
Then, \(|g (x + u)| < |g (x)| + \varepsilon\) and combining this with the bound
in the above display,
\begin{align*}
  |g (x + u)| \left| \kappa_{h} (u) \right| \leq
  & \ (|g (x)| + \varepsilon) \left| \kappa_{h} (u) \right|
  \cdot \mathbf{1} \left\{ \|u\| < \delta \right\} + |g (x + u)| \left|
  \kappa_{h} (u) \right| \cdot \mathbf{1} \left\{ \|u\| \geq \delta \right\} \\
  \leq
  & \ (|g (x)| + \varepsilon) \left| \kappa_{h} (u) \right| \cdot \mathbf{1}
  \left\{ \|u\| < \delta \right\} \\
  & + \frac{|g (x + u)|}{\|u\|^{d}} \|u\|^{d} \left| \kappa_{h} (u) \right|
  \cdot \mathbf{1} \left\{ \|u\| \geq \delta \right\} \\
  \leq
  & \ (|g (x)| + \varepsilon) \left| \kappa_{h} (u) \right| \cdot \mathbf{1}
  \left\{ \|u\| < \delta \right\} \\
  & + \frac{|g (x + u)|}{\delta^{d}} \|u\|^{d} \left| \kappa_{h} (u) \right|
  \cdot \mathbf{1} \left\{ \|u\| \geq \delta \right\} \\
  \leq
  & \ (|g (x)| + \varepsilon) \left| \kappa_{h} (u) \right| \cdot \mathbf{1}
  \left\{ \|u\| < \delta \right\} \\
  & + \frac{1}{\delta^{d}} \left[ \sup_{y \in \mathbb{R}^{d} : \|y\| \geq
  \delta} \|y\|^{d} \left| \kappa_{h} (y) \right| \right] \cdot |g (x + u)|
  \mathbf{1} \left\{ \|u\| \geq \delta \right\}.
\end{align*}
The supremum in the above display exists and is finite for \(h\) sufficiently
small by \eqref{eqn--conv-kernel-thin-tail-condition}.
We know that \(\left| \kappa_{h} (u) \right|\) has a finite integral by
\Cref{asm--convl-kernel} \ref{asm--convl-kernel-L1}.
Therefore, the a sufficient condition for the existence and finiteness of the
integral in \eqref{eqn--convl-kernel-approximant} is integrability of \(|g (x +
u)|\) since then \(|g (x + u)| \mathbf{1} \{\|u\| \geq \delta\}\) would be
integrable.
To that end, note that \(\int |g (x + u)| \; \mathrm{d} u = \int |g (v)| \;
\mathrm{d} v < \infty\) by the change of variables \(v = x + u\).

To prove the limit claims, we show \eqref{eqn--convl-kernel-conv-criteria} in
\Cref{lem--convl-kernel-conv-criteria}.
Given any \(\delta > 0\),
\begin{align*}
  \int_{\|u\| \geq \delta} |g (x + u)| \left| \kappa_{h} (u) \right| \;
  \mathrm{d} u =
  & \, \int_{\|u\| \geq \delta} \frac{|g (x + u)|}{\|u\|^{d}} \cdot \|u\|^{d}
  \cdot \left| \kappa_{h} (u) \right| \; \mathrm{d} u \\
  =
  & \, \left[ \sup_{y \in \mathbb{R}^{d} : \|y\| \geq \delta} \|y\|^{d} \cdot
  \left| \kappa_{h} (y) \right| \right] \cdot \int_{\|u\| \geq \delta} \frac{|g
  (x + u)|}{\|u\|^{d}} \; \mathrm{d} u \\
  \leq
  & \, \frac{1}{\delta^{d}} \left[ \sup_{y \in \mathbb{R}^{d} : \|y\| \geq
  \delta} \|y\|^{d} \cdot \left| \kappa_{h} (y) \right| \right] \cdot
  \int_{\|u\| \geq \delta} |g (x + u)| \; \mathrm{d} u.
\end{align*}
In addition,
\begin{equation*}
  \int_{\|u\| \geq \delta} |g (x + u)| \; \mathrm{d} u = \int_{\|x - v\| \geq
  \delta} |g (v)| \; \mathrm{d} v \leq \int |g (v)| \; \mathrm{d} v.
\end{equation*}
Combining both bounds,
\begin{equation*}
  \int_{\|u\| \geq \delta} |g (x + u)| \left| \kappa_{h} (u) \right| \;
  \mathrm{d} u \leq \frac{1}{\delta^{d}} \left[ \sup_{y \in \mathbb{R}^{d} :
  \|y\| \geq \delta} \|y\|^{d} \cdot \left| \kappa_{h} (y) \right| \right] \cdot
  \int |g (v)| \; \mathrm{d} v.
\end{equation*}
Given any \(\delta > 0\), the right hand side of the above display tends to 0 as
\(h \to 0\) by \eqref{eqn--conv-kernel-thin-tail-condition}.
Both limiting claims in \Cref{thm--convl-kernel-conv-thin-tail} now follow from
\eqref{eqn--convl-kernel-conv-criteria} in
\Cref{lem--convl-kernel-conv-criteria}.
\end{proof}

\begin{theorem}
\label{thm--convl-kernel-eg-thin-tail}
Let \(\kappa : \mathbb{R}^{d} \to \mathbb{R}\) be an integrable function with
\(\lim_{\|y\| \to \infty} \|y\|^{d} \kappa (y) = 0\).
Let \(x \in \mathbb{R}^{d}\), and let \(g : \mathbb{R}^{d} \to \mathbb{R}\) be
an integrable function that is continuous at \(x\).
Then
\begin{equation}
  \lim_{h \to 0} \int g (x + v h) \kappa (v) \; \mathrm{d} v = \lim_{h \to
  0} \frac{1}{h^{d}} \int g (x + u) \kappa (u / h) \; \mathrm{d} u =
  g (x) \int \kappa (u) \; \mathrm{d} u.
  \label{eqn--convl-kernel-eg-thin-tail}
\end{equation}
\end{theorem}

\begin{proof}[Proof of \Cref{thm--convl-kernel-eg-thin-tail}]
Let \(\kappa_{h} (u) := h^{- d} \kappa (u / h)\).
The first equality in \eqref{eqn--convl-kernel-eg-thin-tail} follows from
the change of variables \(v = u / h\).
So we prove the second equality.
By \Cref{lem--convl-kernel-eg} and \Cref{lem--convl-kernel-eg-thin-tail},
\(\left\{ \kappa_{h} \right\}\) satisfies \Cref{asm--convl-kernel} and
\Cref{asm--convl-kernel-conv-thin-tail}.
By the change of variables \(v = u / h\), \(\underline{\kappa}_{h} = h^{- d}
\int \kappa (u / h) \; \mathrm{d} u = \int \kappa (v) \; \mathrm{d} v\) for
every \(h \in (0, \infty)\).
Hence, \(\lim_{h \to 0} \underline{\kappa}_{h} = \underline{\kappa} := \int
\kappa (v) \; \mathrm{d} v\) is trivially true.
Then \eqref{eqn--convl-kernel-eg-thin-tail} follows from
\Cref{thm--convl-kernel-conv-thin-tail}.
\end{proof}

%%% Local Variables:
%%% mode: LaTeX
%%% TeX-master: "../note-kdre-inference"
%%% End:


% LocalWords:  cdf ecdf
