%! TEX root = ../note-kdre-inference.tex

\section{Approximation Results for Kernel Estimators}
% \label{sec--kernel-approx}

Here, we consider approximation by convolution.
The exposition here is mainly based on
\citet[pp. 362--365]{1999paganNonparametricEconometrics}.%
\footnote{In my PDF copy of \citet{1999paganNonparametricEconometrics}, this
corresponds to pp. 381--384 (in PDF).}
I should note that \citet{1999paganNonparametricEconometrics} borrow from
\citet{1962parzenEstimationProbabilityDensity} who in turn borrows from
\citet{1955bochnerHarmonicAnalysisTheory}.

\begin{assumption}
\label{asm--convolution-kernel}
\(\left\{ K_{h} : h \in (0, \infty) \right\}\) is a family of
Lebesgue-integrable functions mapping \(\mathbb{R}^{d}\) to \(\mathbb{R}\)
satisfying
For \(h \in (0, \infty)\), \(K_{h} : \mathbb{R}^{d} \to \mathbb{R}\)
\end{assumption}

For each \(h \in (0, \infty)\), let \(K_{h} : \mathbb{R}^{d} \to \mathbb{R}\) be
a Borel-measurable and Lebesgue-integrable function.
We wish to approximate a function \(g : \mathbb{R}^{d} \to \mathbb{R}\) using
at least pointwise at points of continuity


%%% Local Variables:
%%% mode: LaTeX
%%% TeX-master: "../note-kdre-inference"
%%% End:

% LocalWords:  cdf ecdf
