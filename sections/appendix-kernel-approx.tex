%! TEX root = ../note-kdre-inference.tex

\section{Approximation Results for Kernel Estimators}
% \label{sec--kernel-approx}

Here, we consider approximation by convolution.
The exposition here is mainly based on
\citet[pp. 362--365]{1999paganNonparametricEconometrics}.%
\footnote{In my PDF copy of \citet{1999paganNonparametricEconometrics}, this
corresponds to pp. 381--384 (in PDF).}
I should note that \citet{1999paganNonparametricEconometrics} borrow from
\citet{1962parzenEstimationProbabilityDensity} who in turn borrows from
\citet{1955bochnerHarmonicAnalysisTheory}.

\begin{assumption}
\label{asm--convl-kernel}
\(\left\{ K_{h} : h \in (0, \infty) \right\}\) is a family of
functions mapping \(\mathbb{R}^{d}\) to \(\mathbb{R}\)
satisfying the following.
\begin{enumerate}[label=(\roman*)]
  \item \label{asm--convl-kernel-L1}
    For each \(h \in (0, \infty)\), \(K_{h}\) is Lebesgue-integrable with
    \(\int \left| K_{h} (u) \right| \; \mathrm{d} u =: \overline{K}_{h} <
    \infty\).
  \item \label{asm--convl-kernel-integral-unit}
    For each \(h \in (0, \infty)\), \(\int K_{h} (u) \; \mathrm{d} u = 1\).
  \item \label{asm--convl-kernel-integral-unif-bound}
    There is a \(\overline{K} \in (0, \infty)\) such that for every \(h \in (0,
    \infty)\), \(\overline{K}_{h} \leq \overline{K}\).
  \item \label{asm--convl-kernel-convl-prob}
    For each \(\delta \in (0, \infty)\), \(\lim_{h \to 0} \int_{\|u\| \geq
    \delta} \left| K_{h} (u) \right| \mathrm{d} u = 0\).
\end{enumerate}
\end{assumption}

\begin{remark}
% \label{rem--convl-kernel-non-neg}
If the family \(\left\{ K_{h} \right\}\) satisfies \(K_{h} \geq 0\) almost
everywhere for each \(h \in (0, \infty)\), then \Cref{asm--convl-kernel}
\ref{asm--convl-kernel-L1} and \ref{asm--convl-kernel-integral-unit} imply
\ref{asm--convl-kernel-integral-unif-bound} with \(\overline{K} = 1\), since in
this case \(\overline{K}_{h} = 1\) for every \(h \in (0, \infty)\).
\end{remark}

\begin{theorem}
\label{lem--convl-kernel-eg}
Let \(K : \mathbb{R}^{d} \to \mathbb{R}\) be a Lebesgue-integrable function with
\(\int K (u) \; \mathrm{d} u = 1\).
Define \(K_{h} (u) := h^{- d} K (u / h)\).
Then \(\left\{ K_{h} \right\}\) satisfies \Cref{asm--convl-kernel}.
\end{theorem}

\begin{proof}[Proof of \Cref{lem--convl-kernel-eg}]
By the change of variables \(v = u / h\),
\begin{align*}
  & \int K_{h} (u) \; \mathrm{d} u = \int h^{- d} K (u / h) \; \mathrm{d} u =
  \int K (v) \; \mathrm{d} v = 1, \\
  & \overline{K}_{h} = \int \left| K_{h} (u) \right| \; \mathrm{d} u = \int h^{-
  d} |K (u / h)| \; \mathrm{d} u = \int |K (v)| \; \mathrm{d} v =: \overline{K},
  \\
  & \int_{\|u\| \geq \delta} \left| K_{h} (u) \right| \; \mathrm{d} u =
  \int_{\|u\| \geq \delta} h^{- d} \left| K (u / h) \right| \; \mathrm{d} u =
  \int_{\|v\| \geq \delta / h} |K (v)| \; \mathrm{d} v \to 0 \\
  & \text{since} \quad \{v : \|v\| \geq \delta / h\} \searrow \emptyset \quad
  \text{as} \quad h \to 0.
\end{align*}
\end{proof}

For a measurable function \(g : \mathbb{R}^{d} \to \mathbb{R}\),
consider the convolution based approximant
\begin{equation}
  \begin{gathered}
  g_{h} (x) := T_{h} [g] (x) := \int g (x - u) K_{h} (u) \; \mathrm{d} u \\
  \text{for a family } \left\{ K_{h} \right\} \text{ satisfying
  \Cref{asm--convl-kernel}.}
  \end{gathered}
  \label{eqn--convl-kernel-approximant}
\end{equation}
Assume henceforth that for the point \(x\) in question, the integral in
\eqref{eqn--convl-kernel-approximant} exists and is finite.
We first provide a pointwise error bound for \(g_{h} - g\).

\begin{theorem}
\label{thm--convl-kernel-point-error-bound}
Suppose the family \(\left\{ K_{h} \right\}\) satisfies
\Cref{asm--convl-kernel}.
Let \(h, \delta \in (0, \infty)\), \(x \in \mathbb{R}^{d}\) be given, and let
\(g : \mathbb{R}^{d} \to \mathbb{R}\) be a measurable function.
Assume the integral in \eqref{eqn--convl-kernel-approximant} exists and is
finite.
Then
\begin{equation}
  \begin{split}
    \left| g_{h} (x) - g (x) \right| \leq
    & \, \left\{ \sup_{\|u\| < \delta} |g (x - u) - g (x)| \right\} \cdot
    \overline{K} + \int_{\|u\| \geq \delta} |g (x - u)| \cdot \left| K_{h} (u)
    \right| \; \mathrm{d} u \\
    & + |g (x)| \int_{\|u\| \geq \delta} \left| K_{h} (u) \right| \; \mathrm{d}
    u.
  \end{split}
  \label{eqn--convl-kernel-inequality-main}
\end{equation}
\end{theorem}

\begin{proof}[Proof of \Cref{thm--convl-kernel-point-error-bound}]
By \(\int K_{h} (u) \; \mathrm{d} u = 1\),
\begin{equation*}
  \left| g_{h} (x) - g (x) \right| = \left| \int (g (x - u) - g (x)) K_{h} (u)
  \; \mathrm{d} u \right| \leq \int |g (x - u) - g (x)| \left| K_{h} (u) \right|
  \; \mathrm{d} u.
\end{equation*}
Split the integral on the right:
\begin{align*}
  \left| g_{h} (x) - g (x) \right| \leq
  & \, \int_{\|u\| < \delta} |g (x - u) - g (x)| \left| K_{h} (u) \right| \;
  \mathrm{d} u % \\ &
  + \int_{\|u\| \geq \delta} |g (x - u) - g (x)| \left| K_{h} (u) \right| \;
  \mathrm{d} u \\
  \leq
  & \, \sup_{\|u\| < \delta} |g (x - u) - g (x)| \int \left| K_{h} (u) \right|
  \; \mathrm{d} u \\
  & + \int_{\|u\| \geq \delta} |g (x - u)| \left| K_{h} (u) \right| \;
  \mathrm{d} u + |g (x)| \int_{\|u\| \geq \delta} \left| K_{h} (u) \right| \;
  \mathrm{d} u.
\end{align*}
In the last inequality above, bound the integral in the first term by
\(\overline{K}\) to get \eqref{eqn--convl-kernel-inequality-main}.
\end{proof}

\begin{lemma}
\label{lem--convl-kernel-conv-criteria}
Suppose the family \(\left\{ K_{h} \right\}\) satisfies
\Cref{asm--convl-kernel}.
Let \(x \in \mathbb{R}^{d}\) be given, and let \(g : \mathbb{R}^{d} \to
\mathbb{R}\) be a measurable function that is continuous at \(x\).
Assume the integral in \eqref{eqn--convl-kernel-approximant} exists and is
finite for \(h > 0\) sufficiently small.
\begin{equation}
  \text{If} \quad
  \forall \delta > 0, \lim_{h \to 0} \int_{\|u\| \geq \delta} |g
  (x - u)| \left| K_{h} (u) \right| \; \mathrm{d} u = 0,
  \quad \text{then} \quad
  \lim_{h \to \infty} g_{h} (x) = g (x).
  \label{eqn--convl-kernel-conv-criteria}
\end{equation}
\end{lemma}

\begin{proof}[Proof of \Cref{lem--convl-kernel-conv-criteria}]
Since \(g\) is continuous at \(x\), the first term in
\eqref{eqn--convl-kernel-inequality-main} can be controlled by choice of
\(\delta\) sufficiently small.
For any choice of \(\delta\), the second term in
\eqref{eqn--convl-kernel-inequality-main} is controlled by choice of \(h\)
sufficiently small by
\eqref{eqn--convl-kernel-conv-criteria}.
For any choice of \(\delta\), the third term in
\eqref{eqn--convl-kernel-inequality-main} is controlled by choice of \(h\)
sufficiently small due to \Cref{asm--convl-kernel}
\ref{asm--convl-kernel-convl-prob}.
\end{proof}

We therefore need some way(s) to deal with the second term.
We present two ways to do this.

\begin{theorem}
\label{thm--convl-kernel-conv-bounded}
Suppose the family \(\left\{ K_{h} \right\}\) satisfies
\Cref{asm--convl-kernel}.
Let \(x \in \mathbb{R}^{d}\) be given, and \(g : \mathbb{R}^{d} \to \mathbb{R}\)
be a bounded and measurable function that is continuous at \(x\).
Then the integral in \eqref{eqn--convl-kernel-approximant} exists and is finite
and furthermore, \(\lim_{h \to 0} g_{h} (x) = g (x)\).
\end{theorem}

\begin{proof}[Proof of \Cref{thm--convl-kernel-conv-bounded}]
Existence and finiteness of the integral in
\eqref{eqn--convl-kernel-approximant} are immediate consequences of the
hypothesis of \(g\) being bounded.
Furthermore by this hypothesis, given any \(\delta > 0\)
\begin{equation*}
  \int_{\|u\| \geq \delta} |g (x - u)| \left| K_{h} (u) \right| \; \mathrm{d} u
  \leq \sup_{y \in \mathbb{R}^{d}} |g (y)| \int_{\|u\| \geq \delta}
  \left| K_{h} (u) \right| \; \mathrm{d} u \to 0 \quad \text{as} \quad h \to 0.
\end{equation*}
By \eqref{eqn--convl-kernel-conv-criteria} in
\Cref{lem--convl-kernel-conv-criteria}, \(\lim_{h \to 0} g_{h} (x) = g (x)\) as
desired.
\end{proof}

\begin{assumption}
\label{asm--convl-kernel-conv-thin-tail}
The family \(\left\{ K_{h} \right\}\) in
\Cref{asm--convl-kernel} also satisfies the following condition:
\begin{equation}
  \forall \delta > 0, \quad
  \lim_{h \to 0} \sup_{\|u\| \geq \delta} \|u\|^{d} K_{h} (u) = 0.
  \label{eqn--conv-kernel-thin-tail-condition}
\end{equation}
\end{assumption}

\begin{remark}
Consider the function \(K\) and the associated family \(\left\{ K_{h} \right\}\)
defined in \Cref{lem--convl-kernel-eg}.
Condition \eqref{eqn--conv-kernel-thin-tail-condition} is satisfied if
\begin{equation}
  \lim_{\|y\| \to \infty} \|y\|^{d} K (y) = 0.
  \label{eqn--conv-kernel-thin-tail-condition-1}
\end{equation}
To see this, note that
\begin{align*}
  \sup_{\|u\| \geq \delta} \|u\|^{d} K_{h} (u) = \sup_{\|u\| \geq \delta} \| u /
  h \|^{d} K (u / h),
\end{align*}
and given any \(\delta > 0\), \eqref{eqn--conv-kernel-thin-tail-condition-1}
ensures that the right hand side above can be made arbitrarily small when \(h
\to 0\).
\end{remark}

\begin{theorem}
\label{thm--convl-kernel-conv-thin-tail}
Suppose the family \(\left\{ K_{h} \right\}\) satisfies
\Cref{asm--convl-kernel} and \Cref{asm--convl-kernel-conv-thin-tail}.
Let \(x \in \mathbb{R}^{d}\) be given, and let \(g : \mathbb{R}^{d} \to
\mathbb{R}\) be a Lebesgue-integrable function that is continuous at \(x\).
Assume also that the integral in \eqref{eqn--convl-kernel-approximant} exists
for all \(h > 0\) sufficiently small.
Then \(\lim_{h \to 0} g_{h} (x) = g (x)\).
\end{theorem}

\begin{proof}[Proof of \Cref{thm--convl-kernel-conv-thin-tail}]
Given any \(\delta > 0\)
\begin{align*}
  \int_{\|u\| \geq \delta} |g (x - u)| \left| K_{h} (u) \right| \; \mathrm{d} u
  =
  & \, \int_{\|u\| \geq \delta} \frac{|g (x - u)|}{\|u\|^{d}} \cdot \|u\|^{d}
  \cdot \left| K_{h} (u) \right| \; \mathrm{d} u \\
  =
  & \, \left[ \sup_{y \in \mathbb{R}^{d} : \|y\| \geq \delta} \|y\|^{d} \cdot
  \left| K_{h} (y) \right| \right] \cdot \int_{\|u\| \geq \delta} \frac{|g (x -
  u)|}{\|u\|^{d}} \; \mathrm{d} u \\
  \leq
  & \, \frac{1}{\delta^{d}} \left[ \sup_{y \in \mathbb{R}^{d} : \|y\| \geq
  \delta} \|y\|^{d} \cdot \left| K_{h} (y) \right| \right] \cdot \int_{\|u\|
  \geq \delta} |g (x - u)| \; \mathrm{d} u.
\end{align*}
In addition,
\begin{equation*}
  \int_{\|u\| \geq \delta} |g (x - u)| \; \mathrm{d} u = \int_{\|x - v\| \geq
  \delta} |g (v)| \; \mathrm{d} v \leq \int |g (v)| \; \mathrm{d} v.
\end{equation*}
Combining both bounds,
\begin{equation*}
  \int_{\|u\| \geq \delta} |g (x - u)| \left| K_{h} (u) \right| \; \mathrm{d} u
  \leq \frac{1}{\delta^{d}} \left[ \sup_{y \in \mathbb{R}^{d} : \|y\| \geq
  \delta} \|y\|^{d} \cdot \left| K_{h} (y) \right| \right] \cdot \int |g (v)| \;
  \mathrm{d} v.
\end{equation*}
Given any \(\delta > 0\), the right hand side of the above display tends to 0 as
\(h \to 0\) by \eqref{eqn--conv-kernel-thin-tail-condition}.
By \eqref{eqn--convl-kernel-conv-criteria} in
\Cref{lem--convl-kernel-conv-criteria}, \(\lim_{h \to 0} g_{h} (x) = g (x)\) as
desired.
\end{proof}

%%% Local Variables:
%%% mode: LaTeX
%%% TeX-master: "../note-kdre-inference"
%%% End:

% LocalWords:  cdf ecdf
